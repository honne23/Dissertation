\section{Related Work}
In this section I will explore and contrast current computational approaches to the PSP problem,
including those that operate over lattices (on-lattice) and relevant techniques and results from
off-lattice prediction methods.
\subsection{MCMC \& Integer Programming methods for lattice models}
Markov Chain Monte-Carlo (MCMC) is a random sampling technique that aims to parameterise
an approximation of the posterior distribution of a given dataset. Data points are sequentially sampled from the target distribution
using arbitary parameters $\theta_1 , \theta_2, \hdots, \theta_n$, the distribution is then approximated
using gradient descent (or hill-climbing) by comparing the ratios of the likelihoods of the posterior 
for each succesive pairs of parameters $\frac{p(data | \theta_2)}{p(data | \theta_1)}$. The parameters of the approximating
distribution are then updated using this ratio, often in addition to a standard learning 
rate $\eta$; this is effectively an optimization in parameter space. The metropolis-random walk algorithm
updates the parameters based on an acceptance criterion. Canonically, a random number is uniformly sampled
on the interval $[0,1]$ and the parameter update is "accepted" if the ratio is greater than the random number.
The sequential sampling produces auto-correlated samples, indicating that each data-point sampled is
probabilistically dependent on the previous point as a function of time; this gives rise to the \emph{markov property}
of the sampling technique. Under this construction, the steady state (equilibrium) distribution of the chain\footnote{See equation 2.14}
is equal to the posterior under investigation; a more thorough analysis is out of the scope of this project however
the reader is directed to \cite{levin2017markov}. \\

\cite{Hansmann1999} gives a detailed overview of markov chain monte-carlo (MCMC) methods and genetic
algorithms for structure prediction. Previous approaches to structure prediction generally involved stochastic
approximation methods such as MCMC in order to tackle the curse of dimensionality when working
with atomic parameters. This often included the use of genetic algorithms
to generate candidate structures and crystal growth simulations where MCMC to randomly
place a residue (obeying SAW constraints) in unrestricted 3D space followed by local optimization
on the structure. In addition to this, if parameter updates are accepted according to a \emph{Boltzmann}
weighting, simulated logarithmic annealing schedules were used to modulate the temperature parameter, often improving results. 
The author notes that although this was effective in smaller peptides, the approaches generally
did not scale to larger structures. Furthermore, the resultant structures would often lack detailed balance;
although the global structure was approximately optimal, residues in local structures would form unfavourable
contacts. \\

\cite{Citrolo2013} describe a hybrid approach between previously mentioned MCMC methods and ant colony optimization (ACO)
using the HP model on a 3D cubic lattice (see eq. 2.7).
ACO is a biologically inspired heuristic search algorithm where each "ant" represents a possible solution, these
agents communicate locally to coordinate their next move towards a more optimal position in parameter space.
The authors present a modified version of the objective function described in table 2.2 that penalizes
overlapping walks. As in previous work, they utilise MCMC to refine a global objective function, and then
use the ants to optimize local solutions. The communication bias between the agents appeared to encourage
better results as compared with previous methods. Once again, this proved to be effective on smaller peptides
but scaling would still be an issue. An insightful observation by the authors suggested that the existence of overlapping
solutions resulted in a rough fitness lanscape, reducing stability during the training phase. \\

\cite{Yanev2017} similarly proposed a mixed integer programming solution to the HP model, once again
on a 3D cubic lattice. Mixed integer programming (MIP) is an optimization method that optimizes constraints
whose requirements are represented by linear relationships, in this particular case some of the constraints
are expressed as integers whereas others may take on real values. The authors evaluate PSP as a bi-partite
graph matching problem on local neighbourhoods, taking into account the SAW. A bi-partite graph in one whose
vertices can be divided into two disjoint and independent sets \cite{Handa1999BipartitieGW}, an example is
provided:

\definecolor{myblue}{RGB}{80,80,160}
\definecolor{mygreen}{RGB}{80,160,80}
\begin{figure}[!htb]
\begin{center}
    \begin{tikzpicture}[thick,
        every node/.style={draw,circle},
        fsnode/.style={fill=myblue},
        ssnode/.style={fill=mygreen},
        every fit/.style={ellipse,draw,inner sep=-2pt,text width=2cm},
        ->,shorten >= 3pt,shorten <= 3pt
      ]
      
      % the vertices of U
      \begin{scope}[start chain=going below,node distance=7mm]
      \foreach \i in {1,2,...,5}
        \node[fsnode,on chain] (f\i) [label=left: \i] {};
      \end{scope}
      
      % the vertices of V
      \begin{scope}[xshift=4cm,yshift=-0.5cm,start chain=going below,node distance=7mm]
      \foreach \i in {6,7,...,9}
        \node[ssnode,on chain] (s\i) [label=right: \i] {};
      \end{scope}
      
      % the set U
      \node [myblue,fit=(f1) (f5),label=above:$U$] {};
      % the set V
      \node [mygreen,fit=(s6) (s9),label=above:$V$] {};
      
      % the edges
      \draw (f1) -- (s6);
      \draw (s6) -- (f2);
      \draw (f2) -- (s7);
      \draw (s7) -- (f3);
      \draw (s8) -- (f3);
      \draw (f3) -- (s9);
      \draw (s9) -- (f5);
      \draw (f5) -- (s6);
      
      \end{tikzpicture}  
\end{center}
\caption{A bi-partite graph}
\end{figure}
\cite{Yanev2017} succesively join adjacent sites by optimizing constraints on bi-partite graphs in local neighbourhoods
on the lattice using MIP, their methods suffer from the same drawbacks as previously listed.
\subsection{Deep Learning methods for protein structure prediction}
Deep learning methods surrounding PSP generally follow the same approach; predicting properties
such as inter-residue distances or torsion angles from a dataset of proteins and their known tertiary structures,
effectively modelling the problem as a supervised learning task which can be optimised using a differentiable loss 
function with gradient descent.
The most effective contemporary approaches all share a core structure, a Convolutional Neural Network (CNN)
with residual connections (ResNet: \cite{He2016}):

A CNN takes in a matrix $x \in \mathbb{R}^n \times \mathbb{R}^n$ as input with an arbitary number of channels.
In image recognition, the input is three matrices of the image resolution (e.g 1920 x 1080), one for each
colour channel RGB. A square matrix (usually $3 \times 3$ \emph{kernel}) is initialised to arbitary values, then slid over each
input channel, the element-wise product is summed over the overlapping components to produce a scalar value that
is taken as input again in the next layer where the process is repeated. The elements of the kernel are parameters 
learned by the network and multiple such kernels can be learned. The inner-most layers of the network form high level
representations of the input data that can be utilised for computation by the output layers. \cite{Hou2019,Gao2019,Senior2020,Yang2020}
all use feature maps derived from MSA analysis to produce multiple channels as input to the CNN. The aim here 
is to exploit the spatial coherance bias imposed by the sliding kernels that act on local neighbourhoods. 
As depicted in the figure, the output is a matrix of contact maps although the exact nature of this varies among 
the author's approaches. A contact map $a_{ij}$ for $n$ residues is a $n \times n$ matrix whose components encode
the distances between residues at indices $i,k$ in Angstroms $\mathring{A}$, an atomic unit of measurement.
\subsubsection{Alpha-Fold}
It is necessary to bring particular attention to the results of AlphaFold \cite{Senior2020} due
to their impressive results at the Critical Assesment for Structure Prediction 13 (CASP13) competition.
First place was awared for the work of \cite{Senior2020}; beating second by a large margin when they correctly
predicted the structures of 23/45 proteins to significant accuracy, which had not been previously released to the public.
Their pipeline consists of input data derived from MSA features and contact map data, this is passed into
a ResNet as described previously, however their output is a 3D tensor. This tensor encodes the contacts
of indices $i,j$ along the 2-dimensional axis, and the 3rd dimension represents a distribution over possible
inter-residue distances at that contact. This is similar to the previously explore \textbf{Distributional Learning}, where a
distribution of returns is estimate for each action; in this case a distribution is approximated using
64 bins for every contact $i,j$ which they call a \emph{distogram}. This distogram is then used to inform
a differentiable model of the protein's specific potential (energy function) parameterised by its torsion
angles $(\phi, \psi)$. The precise implementation is out of the scope of this work, however is it important to note 
that the authors attribute the success of their model to its particular ability to predict \emph{distributions}, arguing
that this allowed them to generate much more plausible lowest-energy conformation candidate structures during an ensemble
prediction phase. Intuitively, this is inline with the biological consensus that proteins exist as a 
specific distribution of strutures in their native tertiary form \cite{Yang}, and so by incorporating the calculation
of distributions into their model, they implicitly embed this prior in the architecture of their model (see Section \textbf{2.1.3}).\\

\cite{Yang2020} build on the success of AlphaFold and improve upon their results by additionally predicting
distributions of inter-residue orientations and incorporating that into the model of the protein's specific potential.
\subsection{Deep Reinforcement Learning methods for lattice models}
We now turn to reinforcement learning methods on lattice models. Where as previous methods relied on
training data in order to parameterise a generative model, these works learn to construct tertiary structures
from raw sequence data alone, this is unsupervised learning.
\cite{Wu2019} use reinforcement learning on a 2D HP lattice to construct proteins by iteratively
placing residues on points on the lattice using $Q$ learning. Although the authors make no mention of
deep learning in particular,they use function approximation to represent the Q values. In their setting,
the agent has access to a "bag" of residues that must be placed on points in the lattice in the appropriate
sequence, where the state that is input to the agent is the current coordinates of the placed residues. 
The actions available to the agent are movement vectors that determine the next location of the residue on the 
lattice, the agent is rewarded for making favourable contacts along the way, and when the terminal state is reached
the total reward for the whole structure is provided. They describe a \emph{rigid selection criterion}, where the agent
repeatedly selects an action until a legal action is chosen, one that does not violate the SAW. The authors note that having access
to the full set of actions at anytime allowed the network to form a better description of the energy landscape. They 
use the reward structure of the HP model as a proxy for the energy function (or \emph{Hamiltonian}), similar to the 
protein specific potential described by \cite{Senior2020}. They also describe a \emph{flexible} selection criterion
that would give the agent a reward of $-10$ for violating the SAW, once the action is taken the agent is placed in the invalid
position and the reward is reset to 0. They show that the rigid criterion was more effective in their results.\\

\cite{Yanjun2018} describe a similar architecture using a 2D HP grid, this time with the explicit use of deep reinforcement learning.
However instead of providing the residue's current coordinates, they provide the entire grid as input to agent. Formally,
the input is a 3D tensor that encoded the occupation status of a site aswell as the type of occupant, other parameters may additionally be encoded
for each coordinate in the grid. They use an on-policy technique called the Actor-Critic architecture \cite{Mnih2016} to optimise the placement
of residues. A regularization constraint is also imposed, called regularised Upper-Confidence-Bound for Trees (R-UCT) for 
additional stability during training, this was shown to improve results. The algorithm they implemented, Advantage Actor Critic (A2C),
is performed asynchronously, \cite{Yanjun2018}'s approach lends itself to parallel compute and scales much more favourably than
previously described methods, with the exception of \cite{Senior2020,Yang2020}.

\subsection{Multi-agent structure prediction methods}
Turning now to multi-agent approaches to PSP, these methods use agents that collaborate
amongst themselves in a variety of schemes in an effort to optimise a global goal, usually the protein's specific
potential, by building upon local interactions. This is a more biologically consistent view of the micro-level
interactions between the residues. There is limited work at the intersection between the fields of multi-agent learning
and structure prediction, the most relevant works are presented here.\\

 \cite{Muscalagiu2013,Czibula2011} utilize agents in a distributed contraint optimization setting is similar fashion to previously described MIP
 approaches. Each residue is an agent, and agents are of heterogenous types, together they colaborate to take
 actions to move on the lattice in order to optimize mutual constraints between them,
 \cite{Muscalagiu2013} use the hHPNX reward structure on 3D/2D cubic and triangular lattice and \cite{Czibula2011} uses a 2D HP lattice.
 They both make use of a distributed variant\footnote{Not to be confused with \emph{distributional}} of $Q$ learning to optimise the constraint parameters. They both
 describe two different types of agents that interact, some agents act as the actual residues, whereas
 a separate "blackboard" agent is used to keep track of all the agents $Q$ values. The agents then 
 incorporate these Q values into their decision policy. Although these techniques are supported in cluster-computing
 environments, the existence of a central agent inhibits scalability, many of these results were performed 
 on small peptides and were inhibited by their inherent time complexity. If every agent must communicate with 
 the blackboard agent to obtain the $Q$ values of all the other agents to complete a full round, then as the number
 of agents grow then this will begin to incur a significant computational cost interms of both network overhead and
 storage requirements. \cite{Czibula2011} suggested the use of function approximation to mitigate these effects;
 these approaches use notably older algorithms with poorer convergence properties compared to current advanced,
 this also played a factor in their final results, reducing the efficacy of an otherwise well structured system.

 \cite{deLimaCorrea2017} use a multi-agent approach to attain extremely compelling results in a free-space environment
 without the use of a discretized lattice. They utilise a hiearchical structure of agents. \emph{Search agents} generate possible
 sub-structures according to a partition of the total peptide assigned to them. They do so using heuristic search strategies 
 discussed previously, the \emph{optimization} agent then performs global optimization, refining the structure
 using evolutionary algorithms and simmulated annealing methods, akin to the crystal growth approaches of MCMC.
 The authors use PDB data to generate a histogram of torsion angles experimentally gathered and this is used to 
 guide the agents towards optimal solutions.