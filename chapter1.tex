%% $Id: chapter1.tex 1790 2010-09-28 16:46:40Z jabriffa $

\chapter{Introduction}

\section{Problem Background}
   Proteins are biological molecules that carry out specific functions within the body,
   they are assembled out of chemical bonds between smaller sub units called amino acid residues
   to form a poly-peptide chain. Proteins carry out their functions 
   by conforming into specific shapes and fitting into substrates that compliment their structure,
   this is referred to as the "lock and key" model of proteins. A
   protein's three-dimensional structure has been proven \cite{Anfinsen} to been
   entirely determined from the ordering of a discrete set of 20 residues
   in a sequence of arbitrary length. The ability to infer a protein's precise\footnote{Within 1\AA}
   three dimensional structure directly from the sequence of residues would
   unlock the potential for designer drugs that carry out specific actions within the body;
   it would also open a path for the treatment of illnesses that arise from malformed
   proteins such as huntington's disease \cite{lesk}.
   This has however been an open problem, as the search space of possible conformations
   for a given protein is vast and subject to numerous local minima. Traditional
   methods of protein structure determination such as X-Ray Crystallography \cite{lesk}
   are extremely costly\footnote{In the order of millions of USD \cite{alberts}} and time
   consuming, sometimes taking up to \emph{four years} to determine the structure of a protein.
   This has given rise to multiple computational approaches \cite{Cymerman2008} as means
   to drive down cost and increase the productivity of researchers. 
   \linebreak
   Naturally, many approaches using deep learning have been proposed to solve the problem,
   recently a breakthrough by DeepMind \textsuperscript{[reference]} propelled them to success
   at the bi-annual Critical Assessment of Structure Prediction (CASP) competition. Though their
   results were state of the art, their methods still relied on building a predictive model of the
   properties of available proteins in the Protein Data Bank (PDB)\textsuperscript{[reference]}; as 
   opposed to inferring the tertiary structure from the primary sequence alone.

\section{Project Aims and Objectives}
   Over the course of this dissertation I aim to explore and contrast
   modern deep learning approaches for approximating the native conformations
   of proteins on a discrete lattice structure. I will then go on to propose
   a novel algorithm based on mean-field approximations that addresses 
   some of the drawbacks and biases inherent in the approaches I have reviewed.
   \linebreak
   The following is a list of the project's aims overall:
   \begin{enumerate}
      \item Provide a succinct introduction to the molecular mechanics that govern the conformations of proteins
      \item Introduce and compare different approaches to modelling the problem computationally
      \item Provide an extended analysis of lattice models for proteins and their ability to encode correct conformations
      \item Introduce reinforcement learning and progress to modern Deep Q-Learning
      \item Building on Deep Q-Learning, introduce improvements to the algorithm since it's inception
      \item Compare and contrast deep learning approaches to the lattice model with an emphasis on reinforcement learning methods
      \item Introduce multi-agent learning within the framework of stochastic games
      \item Introduce mean field game framework
      \item Formulate the conformation of residues on a lattice as a cooperative game of incomplete information
      \item Provide a novel approach to de-novo structure determination using mean-field multi-agent learning
      \item Benchmark my approach against the results of other groups on the same proteins to evaluate the effectiveness of the novel algorithm
   \end{enumerate}
\section{Success Criteria}
   In order to evaluate the utility of both my findings and subsequent algorithm,
   I have defined the following high-level requirements that must be satisfied to 
   mark this project as a success.
   \begin{enumerate}
      \item Provide comprehensive overview of the underlying problem of protein folding
      \item Describe markov decision processes (MDPs) and its ties to reinforcement learning
      \item Highlight drawbacks to the default DQN algorithms and notable improvements to address those
      \item Demonstrate multi-agent learning as a generalisation of single MDPs into markov games
      \item Successfully benchmark my multi-agent approach against similar lattice-based approaches to protein folding
   \end{enumerate}

\section{Structure of Report}
  \begin{enumerate}
      \item \textbf{Introduction} \\
         In this section I provided an overview the the protein folding problem and
         introduced the components that I will be synthesising into novel approach.
      \item \textbf{Literature Review} \\
         Many of the components I introduce throughout this project
         use concepts from multiple fields of literature and much of
         the related work hinges on these topics. In the interest of clarity, 
         I have provided each topic with it's own introduction, their combined
         application is explored in the sub-section \textbf{Related Work}. 
      \item \textbf{System Requirements and Specification} \\
         In this section I will analyse the drawbacks of methods utilized in related work,
         and from this evaluation derive the requirements of a system that
         addresses these limitations.
      \item \textbf{System Design}\\
         This section seeks to unify the selected systems and concepts
         into an integrated learning algorithm that coherently reflects
         the underlying problem's structure. 
      \item \textbf{Testing \& Validation} \\
         In order to verify the efficacy of the learning agents,
         proteins are selected from studies in related work for training
         and the end result is compared to previous work.
      \item  \textbf{Discussions} \\
         Limitations of my implementation are discussed here and
         possible solutions and research directions are proposed.
      \item  \textbf{Conclusion} \\
         Here I will present my concluding thoughts and any 
         additional acknowledges are addressed.
  \end{enumerate}

